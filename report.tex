\documentclass[a4paper,12pt]{report}%extaticle
\raggedbottom
% polyglossia should go first!
\usepackage{polyglossia} % multi-language support

\setdefaultlanguage{russian}
\setmainfont{Times New Roman}
\setsansfont{Times New Roman}
\setmonofont{CMU Typewriter Text}

\setmainlanguage{russian}
\setotherlanguage{english}

\DeclareSymbolFont{letters}{\encodingdefault}{\rmdefault}{m}{it}
\usepackage{amsmath} % math symbols, new environments and stuff
\usepackage{unicode-math} % for changing math font and unicode symbols
\usepackage{bbm}
\setmathfont{XITS Math}

%\usepackage{setspace}
%\doublespacing

\parindent=1.5cm
\usepackage{indentfirst}
\usepackage[left=2cm,right=1cm,top=2cm,bottom=2cm,bindingoffset=1cm]{geometry}% 
%for margins in title page
\renewcommand{\baselinestretch}{1.5}

\usepackage[style=english]{csquotes} % fancy quoting
\usepackage{microtype} % for better font rendering
\usepackage[backend=biber, sorting=none, style=gost-numeric]{biblatex} % for 
%bibliography
\addbibresource{reference_list.bib}

\usepackage{hyperref} % for refs and URLs
\usepackage{graphicx} % for images (and title page)
\usepackage{tabu} % for tabulars (and title page)
\usepackage{placeins} % for float barriers
\usepackage{titlesec} % for section break hooks

\usepackage[labelsep=endash]{caption}
\captionsetup[table]{
    singlelinecheck=false, %table caption per GOST, not centered
    justification=justified}
\captionsetup[figure]{
    name=Рисунок           %picture caption per GOST
}

\usepackage{subcaption} % for subfloats

\usepackage{listings} % for listings
\newfontfamily\listingsfont{Source Code Pro}

\usepackage{enumitem} % for unboxed description labels (long ones)
\usepackage{tikz}     % tikz pictures
\usepackage{rotating} % landscape pictures

\defaultfontfeatures{Mapping=tex-text} % for converting "--" and "---"

\MakeOuterQuote{"} % enable auto-quotation

% new page and barrier after section, also phantom section after clearpage for
% hyperref to get right page.
% clearpage also outputs all active floats:
\newcommand{\sectionbreak}{\clearpage\phantomsection}
\newcommand{\subsectionbreak}{\FloatBarrier}
\newcommand\numberthis{\addtocounter{equation}{1}\tag{\theequation}}
\renewcommand{\thesection}{\arabic{section}} % no chapters
\numberwithin{equation}{section}

\setcounter{tocdepth}{3}

\usepackage{lastpage}
\usepackage[figure,table,xspace]{totalcount}

\usepackage{array}
\newcolumntype{L}[1]{>{\raggedright\let\newline\\\arraybackslash\hspace{0pt}}m{#1}}
\newcolumntype{C}[1]{>{\centering\let\newline\\\arraybackslash\hspace{0pt}}m{#1}}
\newcolumntype{R}[1]{>{\raggedleft\let\newline\\\arraybackslash\hspace{0pt}}m{#1}}

\makeatletter
\define@key{blx@lbx}{fromjapanese}{\blx@defstring{fromjapanese}{#1}}
\define@key{blx@lbx}{langjapanese}{\blx@defstring{langjapanese}{#1}}
\makeatother

\usepackage[final]{pdfpages}
\begin{document}
%title?

\tableofcontents

\section{Введение}
Рождение математики было обусловлено желанием человека познать окружающий мир, обобщить, структурировать и формализовать свое представление о нем~\cite{rudn}. С течением времени математический аппарат стал получать всё большее развитие: на смену формулам площади простых геометрических фигур, выведенных Пифагором и Архимелом, пришло интегральное исчисление Ньютона и Лейбница~\cite{w:integral}, на смену частным случаям поиска корней многочленов второй и третьей степени Виетом~\cite{w:viete} пришел функциональный анализ и линейная алгебра. Неразрешимость сложных проблем механики и физики XVII-XVIII веков~\cite{rudn} привела к созданию математического аппарата решения дифференциальных уравнений, наконец, рождение квантовой физики послужило толчком к упрощению интегрального исчисления и привело к распространению математического аппарата линейных алгебраических операторов.


Получив столь значительное распространение, математика стала развиваться самостоятельно, обобщать и строить абстракции уже на основе собственных теорий и конструкций, и во многом стала работать на опережение запросов к математическом аппарату из других областей научного исследования. Примером этому могут послужить исселования М.~С.~Ли в области матричного исчисления, нашедшие применение в современной компьютерной графике, или исследования Гамильтона в области теории комплексных чисел, широко используемых в современной физике и других областях. Уже в конце XIX -- начале XX века ученым стало понятно, что математика может быть разделена на две дисциплины - фундаментальную и прикладную~\cite{benzi}, и в то время как первая продолжала заниматься развитием собственных математических теорий, вторая посвятила себя реализации этих теорий на практических задачах. Именно о истории и философии прикладой математики пойдет речь в данной работе.


\section{История прикладной математики}
Прикладная (или вычислительная) математика в наиболее близком к современному ее понимаю виде берет свое начало в XX веке сразу в нескольких точках земного шара. Причиной такого широкого распространения является развитие вычислительных мощностей -- появление первых автоматических вычислителей, а также потребность в выполнении сложных расчётов(например баллистичесикх) во времена Второй Мировой войны~\cite{benzi}. Европейская школа вычислительных методов появляется чуть раньше, в Германии и Италии 1920 годов, основными фигурами в которых являются немецкие математики К. Рунге и М. Кутта~\cite{w:runge} и итальянский математик М. Пиконе~\cite{w:picone} -- основатель первого института прикладной математики в Италии. С началом войны, многие ученые становятся вынужденными эмигрировать из Европы, и часть их них оседает в Америке, внося свой вклад в американскую школу этого направления, наиболее яркими представителями которой являются А. Эйнштейн, Д. ф. Нейман.


Среди интересов прикладных математиков тех времен наибольшей популярностью пользуются методы численного решения обыкновенных дифференциальных уравнений(ОДУ) и уравнений в частных производных(УРЧП) и, часто как подзадача, поиск численного решения систем линейных алгебраических уравнений(СЛАУ). В довоенное и военное время становятся открытыми такие методы как метод конечных разностей~\cite{finite_dif_pde}, метод конечных элементов~\cite{cfe_pde}, метод случайных блужданий~\cite{w:random_walk} и критерий устойчивости решения явных схем решения УРЧП Куранта-Фридрихса-Леви~\cite{cfl_pde}.


В связи с тем, что задача получения численного решения СЛАУ является довольно частой подзадачей в методах численного решения ДУ, разработки ученых умов XX века также велись и в этом направлении. Среди наиболее ярких представителей в послевоенные годы можно выделить фон Неймана, с работой направленной на вычисление приближения обратной матрицы~\cite{von_matrix}, а так же А. Тьюринга и Дж. Уилкинсона, К. Ланцо и др. исследователей, стоящих у истоков алгоритмов итерационного решения СЛАУ.


\subsection{Методы решения СЛАУ конца XIX -- середины XX века}
В центре проблемы решения СЛАУ стоит уравнение вида \[Ax = b\]. И если прямые методы решения СЛАУ восходят корнями ещё к древним индийцам~\cite{benzi}, то задачей получения численного решения ученые занялись лишь в начале XIX века -- К.~Ф.~Гаусс~\cite{w:gauss} и его ученик К.~Л.~Герлинг. Причиной подобного интереса послужила необходимость быстрого поиска решений матриц большого размера специального вида, часто возникающих в задачах численного решения различных ДУ и УРЧП.


Существует два подхода к решению СЛАУ: прямой и итерационный. Прямой представляет собой получение точного аналитического решения СЛАУ, что зачастую является неподъемной задачей для современных вычислителей, и создавало такую же проблему для ученых того времени. Помимо вычислительной сложности, классический прямой метод решения СЛАУ на тот момент не учитывал особенностей строения матрицы, в частности разреженность, и не позволял оптимизировать число операций, необходимый для получения численного решения. В условиях использования ``человеческих'' вычислителей~\cite{benzi} в довоенное время это возлагало дополнительные риски, связанные с высокой вероятность получить неправильное решение, напрямую зависящее от числа выполняемых ``вычислителями'' операций.


Альтернативой прямым методам решения СЛАУ являются итерационные. В основе итерационных методов лежит идея получения начального приближения решения слау некоторым образом и его итерационное приближение к искомому. К достоинствам итерационных методов можно отнести то, что на определенных классах задач они могут обеспечить линейную, или близкую к линейной вычислительную сложность, в зависимости от чиста итераций, однако наибольшим недостатком является условная сходимость и зависимость от исходного вида матрицы. 


Идею Гаусса о нахождении итерационного решения СЛАУ в 1845~г. продолжил К.~Г.~Якоби~\cite{w:jacobi}, а затем переоткрыл Ф.~Зейдель~\cite{w:seidel} в 1974~г. В конце XIX века российский ученый Некрасов и итальянский ученый Пизетти усовершентсовали этот метод, доказав зависимость скорости сходимости от величины наибольшего собственного значения. Далее в 1910~г. эту идея получила развитие в методе Ричардсона~\cite{rch_matrix}.


Середина XX века принесла новый виток развития итерационных методов, заключающихся в предобуславливании матрицы исходной задачи и ее трансформации в задачу, вида:
\begin{align*}
	&Ax = b; \\
	&AP^{-1}y = b \\
	&Px = y
\end{align*}
Трансформация задачи таким образом позволяла привести исходную матрицу у кнеобходимому виду, найти промежуточное решение $y$, а затем уже найти искомое решение для $x$. Одним и первых ученых, предложивших такую идею является Л.~Цезари~\cite{cesari_matrix}, исследовавший в своей работе задачу вида $\omega Ax = \omega b, \omega A = B + C$. Идея использования предобуславливателей находит сове применение и в современных методах решения слау и широко освещена во множестве начных статей~\cite{w:preconditioner}.


Альтернативный способ представления постановки задачи решения СЛАУ, послуживший опорой для ряда итерационных методов решения СЛАУ, в середине XX века представил российский математик А.~Н.~Крылов~\cite{w:krylov}. Его идея заключалась в представлении искомого решения слау, как линейной комбинации векторов специального линейного пространства, позже названного пространством Крылова, следующего вида:
\[
	K_n(A, x) = \mathrm{span}\left\{x, Ax, A^2x, \dots, A^{n-1}x\right\}
\]
Искомым решением в таком пространсве является:
\begin{align*}
	&Ax^* = b\\
	&x^* \approx x^{(k)} \in K(A, x^{(0)}),
\end{align*}
где $x^{(0)}$ -- начальное приближение.

\subsection{Современные методы решения СЛАУ}
Идеи современных методов решения СЛАУ лежат в результатах исследования в этой области середины и конца XX века. В середине XX века российским математиком Крыловым классическая


\section{Вывод}


\nocite{*}
\addcontentsline{toc}{section}{Список литературы}
\section*{Список литературы}
\printbibliography[heading=none]
\end{document}